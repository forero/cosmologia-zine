\documentclass[statementpaper,oneside,article,10pt]{memoir}
\usepackage{geometry}
\usepackage{libertine}
\usepackage{kantlipsum}

% Disable chapter/section numbering.
\setsecnumdepth{none}
\maxsecnumdepth{none}

% Optional background
% http://tex.stackexchange.com/a/276280
\usepackage{transparent}
\usepackage{eso-pic}
\newcommand{\BackgroundPic}[1]{%
\put(0,0){%
\parbox[b][\paperheight]{\paperwidth}{%
\vfill
\centering
{\transparent{0.4} \includegraphics[width=\paperwidth,height=\paperheight,%
keepaspectratio]{#1}}%
\vfill
}}}

\begin{document}

% Edit inside the { brackets } to change these.

\title{De las entrañas a la conciencia: un taller de autoindagación cosmológica}
\author{Jaime E. Forero Romero}
\date{Enero 2025}


\begingroup
\let\cleardoublepage\clearpage

%\AddToShipoutPicture*{\BackgroundPic{samplecover}}

\begin{titlingpage}
\maketitle

% Could add a small author's note, etc. here if you like.

\end{titlingpage}

\endgroup

% As the zine is so short, you probably won't need page numbers; however, if you
% want them, comment out the next line with a %.
\pagestyle{empty}


%% CONTENT GOES BELOW


\section{Paso 0: Universos}

La cosmología estudia el orden que le damos al universo. Aquí vamos a expandir este concepto: un universo es cualquier contexto donde podemos observar, interpretar y actuar. Todos vivimos en múltiples universos simultáneamente.

Luego de la meditación guiada, en esta página del cuadernillo, hagan una lista rápida de todos los universos en los que habitan ahora mismo. No piensen demasiado, dejen que las palabras fluyan.

\newpage
\section{Paso 1: Mi lugar en el universo que transformaré}

De todos los universos que listaste, ¿cuál es el que más te gustaría transformar en este momento de tu vida? ¿Por qué?
Márcalo con un círculo y tómate un momento para escribir tu por qué.

Ahora, en este cuadernillo: dibuja el mapa de tu universo elegido, marca tu ubicación actual en él, identifica y nombra los elementos principales. El mapa puede ser abstracto o simbólico. Puedes usar palabras o símbolos. 
\newpage
\section{Paso 2: La mirada profunda}

Ya ubicados en nuestro universo, es momento de observar. Pero no con una mirada superficial, sino con una que se atreve a ver tanto lo visible como lo invisible. Como en la física moderna, que tuvo que aceptar la existencia de la materia oscura y la energía oscura aunque no pudiera verlas directamente.

Responde en este espacio:

1. ¿Qué elementos son visibles a primera vista en este universo?
\vspace{4cm}

2. ¿Qué elementos están ocultos pero sabes que existen?
\vspace{4cm}

3. ¿Qué aspectos te cuesta trabajo observar?
\vspace{4cm}

\newpage

\section{Paso 3: El propósito del universo}

A diferencia del cosmos físico, donde los eventos simplemente suceden sin necesidad de propósito, los universos humanos están llenos de intenciones y significados. Como seres conscientes, constantemente buscamos y creamos sentido.

En cada universo que habitamos hay un origen y un destino, un "de dónde viene todo esto" y un "hacia dónde va". Es momento de explorar el sentido de este universo que has elegido transformar.

Por favor, responde:

1. Sobre el origen:
¿Por qué y por quién existe este universo que estás explorando?  ¿Qué motivó su origen?
\vspace{5cm}

2. Sobre el propósito:
¿Para qué y para quién existe este universo? ¿Qué propósito tiene su evolución?
\vspace{5cm}


\newpage
Algo de texto
\newpage
Algo de texto
\newpage



%% CONTENT ENDS

% Back cover

\newpage

\Large That's it! Have fun.

\end{document}
